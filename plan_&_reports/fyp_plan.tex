\documentclass[]{project_plan}
\bibliographystyle{apalike}
\bibstyle{apalike}
\usepackage{graphicx}
\usepackage{hyperref}
\usepackage{cite}

\newcommand{\bulletPoint}{\hspace{-3.1pt}$\bullet$ \hspace{5pt}}

%---

\def\studentname{Faeq Faisal}
\def\reportyear{2024}
\def\projecttitle{A concurrency based game environment}
\def\supervisorname{Dr Julien Lange}
\def\degree{BSc (Hons) in Computer Science (Software Engineering)}
\def\fullOrHalfUnit{CS3821 Full Unit}
\def\finalOrInterim{Project Plan}

%---

\begin{document}
\maketitle

%---

\chapter*{Declaration}
This plan has been prepared on the basis of my own work. Where other published
and unpublished source materials have been used, these have been acknowledged.
\vskip3em
Word Count: \input{wordcount.txt}words
\vskip3em
Student Name: \studentname
\vskip3em
Date of Submission: October 11, 2024
\vskip3em
Signature:
\vskip0em
\includegraphics[width=3cm]{faeq_faisal_signature}

\newpage

%---

\tableofcontents\pdfbookmark[0]{Table of Contents}{toc}\newpage

%---

\begin{abstract}
  Playing games has become an integral part of modern society, offering not only entertainment but
  also serving as a tool for relaxation and stress relief, while simultaneously fostering critical thinking,
  and problem-solving skills. For certain demographics, such as individuals with disabilities who may have
  limited social interactions, or children still developing, games can provide unique benefits. These include
  learning important social skills like turn-taking, developing emotional intelligence, and building
  resilience~\cite{garaigordobil_developing_2022}. To add, the interactive nature of games, particularly those
  that provided multiplayer experiences, can foster social connections and enhance enjoyment for players;
  "fun experienced when interacting with others is more positive than solitary fun"~\cite{reis_fun_2017}.

  Within multiplayer games, particularly those that support simultaneous play across
  multiple game instances, there is a critical need for leveraging concurrency mechanisms.
  These can be provided by programming languages, runtime environments or other technologies.
  Java, for instance, offers threads that enable resource sharing while maintaining independent execution
  of instructions. An alternative approach is exemplified by the BEAM, a component of the ERTS. The BEAM
  implements concurrency by running schedulers on OS threads, which pull processes from programs - these are
  essentially a set of light-weight processes communicating with each other - making these processes execute in
  \textit{strong isolation} ~\cite{stenman_erlang_2024, armstrong_making_2003, debenedetto_elixir_2019}.
  This abstracts the feature of concurrency from programming languages like Erlang.

  Multiplayer games significantly increase gamers' immersion and enjoyment~\cite{berladean_effect_2016}.
  These concurrency mechanisms enable this by providing a wide range of features crucial for modern
  multiplayer games. These include facilitating simultaneous actions in board games, providing real-time
  error reporting, support for real-time chat systems that run alongside gameplay and dynamic world
  updates across multiple clients. These capabilities, among others, enhance the responsiveness and
  interactivity of multiplayer game environments.

  For implementing the concurrent game environments themselves, there are various architectural approaches.
  To start, there is the choice between the peer-to-peer model or client-server model.
  The peer-to-peer model eliminates the need for a central server, simplifying setup and
  reducing costs. While this model is resilient to server failures, it presents challenges
  in managing peer disconnections and potential cheating through data manipulation~\cite{franchetti_coping_2020}.
  Conversely, the client-server model allows the server to act as a neutral authority to govern the game,
  validating players' actions. The model does introduce more upfront challenges though, including cost,
  latency, and scaling. Mitigations exist, for example, using a load balancer with multiple servers,
  "so that if a server goes down, another one can take its place without causing significant
  disruption for clients", or by having auto-scaling so that servers "can spin up or down
  based on usage" \cite{pandey_peer--peer_2022}.

  To continue, many technologies need to be considered for use in a concurrent environment. For
  real-time communication, solutions include, polling, long polling and websockets. In polling,
  clients send requests to the server regularly, with the server responding with a new message, or
  an empty response, if there isn't one. This has the obvious drawback of keeping load on the server,
  and consuming network bandwidth while there are no new messages. To prevent receiving client requests
  while there are no new messages, in long polling, the server does not send an empty response. Instead,
  it holds the current request until a new message is available or a timeout expires. An issue still remains,
  a connection still has to be kept alive and saved locally on the server. Websockets are a
  solution to both issues. "With web sockets, we can reduce the metadata (HTTP headers) that are sent in
  every request (a shortcoming of Polling) and we can also provide full–duplex communication through a
  single socket (a shortcoming of Long Polling)". This would make websockets the choice for real-time
  communication, however they are not perfect, for e.g., they are still susceptible to DoS attacks
  \cite{gupta_overview_2018}. This is just one of the many use-cases that require an in-depth consideration
  of suitable technologies.

  To produce this project, I will start with exploring different concurrent environment technologies,
  architectures, etc., and make relevant decisions on these. From there, I will develop multiple
  proof-of-concept programs to experiment with the modularity of my chosen technologies, and how they
  can fit together. I will aim to create an engaging and relaxing multiplayer game and an
  environment capable of hosting multiple games simultaneously. It will explore innovative features
  that leverage concurrency to provide experimental gameplay mechanics, such as non-turn-based play.
  Additionally, the project will investigate techniques to mitigate issues like server errors and
  overall system reliability.
\end{abstract}
\newpage

%---

\chapter{Timeline}

For the first term, I will primarily concentrate on research and implementing the core features of the concurrent
environment. It will involve developing the fundamental architecture, integrating various concurrency mechanisms,
and implementing basic gameplay features. The second term will focus on refining the system,
optimizing performance and enhancing user experience. I will try to conduct thorough testing throughout
the project to ensure reliability and scalability of the environment.

\section{Term 1}

\scalebox{1}{
  \begin{tabular}{r |@{\bulletPoint} l}
    Week 1 - 2   & Produce this project plan \& research on viable technologies for use          \\
    Week 3       & Plan final game screens, structure, features, etc.                            \\
    Week 4       & Build the fundamental architecture                                            \\
    Week 5       & Build a proof of concept that exhibits the behaviour of concurrent execution. \\
    Week 6 - 7   & Build a proof of concept board game                                           \\
    Week 8 - 9   & Evaluate proof of concepts \& build the basis of the final game               \\
    Week 10 - 11 & Implement the foundational concurrent features for the final game             \\
  \end{tabular}
}

\section{Term 2}

\scalebox{1}{
  \begin{tabular}{r |@{\bulletPoint} l}
    Week 1 - 2 & Refine and optimize the concurrent features implemented at the end of the first term \\
    Week 3 - 4 & Implement additional game features and polish the user interface                     \\
    Week 5 - 6 & Conduct thorough performance testing and optimize for scalability                    \\
    Week 7 - 8 & Implement improvements to reliability and error handling                             \\
    Week 9     & User testing and gathering feedback                                                  \\
    Week 10    & Address feedback and make final adjustments                                          \\
    Week 11    & Prepare final documentation and project report                                       \\
  \end{tabular}
}

%---

\chapter{Risks and Mitigations}
A concurrent environment introduces unique challenges such as race conditions, deadlocks,
and synchronization issues. These concurrent-specific risks, combined with general project
risks like hardware failures, require careful consideration and mitigation strategies to
ensure the project's success.

\subsubsection{Technical debt}
This is a significant risk, especially when producing a concurrent environment since
the complexity of managing multiple threads / processes can lead to suboptimal
code fairly easily. Quick fixes or improper use of techniques can add up quickly,
resulting in code that is difficult to maintain, debug, or scale. As a result, this risk's
likelihood is high. To mitigate this, I will try to carry out a strict code review process,
focusing on concurrent design patterns and best practices, making sure I refactor any code that needs it.

\subsubsection{Concurrency overhead}
Concurrency can lead to performance issues due to overhead from
context switching, synchronization, and potential bottlenecks, which may
negate the benefits of parallelism, especially due to any lack of knowledge in
concurrent techniques. To mitigate this, I could profile the application to identify
performance bottlenecks and focus on optimizing critical sections. The significance and likelihood
of this risk depends on the technologies I use and their implementations.

\subsubsection{Resource Contention}
This is a large risk since it can prohibit project continuation. When multiple processes
or threads compete for limited resources, it leads to performance issues and potential
deadlocks that prevent the program from continuing. This issue can be difficult to solve,
but it's likelihood is low, especially if I try to mitigate this from the start. I should use
efficient data structures and minimize shared resources to reduce synchronization needs.

\subsubsection{Hardware Failure}
This is a general, low risk where the main hardware, used whilst building the project, fails,
resulting in a loss in data and progress. Since I cannot predict when certain hardware will fail, making
it's likelihood uncertain, I will mitigate the risk of data loss by keeping all my code and reports
under Git version control, using the GitHub (private) and GitLab (private - CIM RHUL) platforms as
remotes and ensure that I push my progress regularly. This means that my work on this project will
be unaffected by hardware failure.

\subsubsection{Poor Estimation of Tasks}
Underestimating the time required for tasks can lead to unrealistic timelines, rushed work and / or failure of
completing all of the planned features, especially when implementing concurrent techniques, which I may
be unfamiliar with. I will try to mitigate this by breaking down the tasks into smaller, manageable components
and use historical data from similar projects online to inform estimates. This is a significant risk with
a high likelihood, especially when taking into consideration the fact that I am a single individual working
on the project, I have multiple other courseworks and deliverables for other modules I take as a student,
that need to be worked on during this project, some of which have deadlines on the same days as this
'final year project' module, and that I also have other personal considerations like how many hours
I wish to work for, during my job.

\subsubsection{Uneven Balance Between Report/Code}
An imbalance between producing the report and code can lead to them hindering each other, making me fall behind
on one or both of them. This is not very likely and has moderate risk, as long as I keep notes as I work. To
mitigate this, I will try to focus on the code first, keeping notes for the write-up, and then start to produce
the report later, giving myself extra time between the change to wrap up unfinished parts.

\subsubsection{Inadequate testing}
Overlooking critical bugs or edge cases due to limited time or resources, can significantly impact the quality
and reliability of the final game. This can lead to risks in progressing on the project like technical debt
and overhead of knowledge in concurrency techniques. This risk is highly likely and has great risk, especially
if I estimate my tasks poorly. To mitigate this, I will conduct thorough testing throughout
the project to ensure it's reliability, as well as carry out the mitigations mentioned for a poor estimation
of tasks.


%-----
\chapter{Acronyms \& Glossary}

Threads

The smallest unit of processing that can be performed in a Java program. Implemented by extending the
Thread class or by implementing the runnable interface.

BEAM

Bogdan's Erlang Abstract Machine / Björn's Erlang Abstract Machine - a virtual machine, built into the OTP
distribution, that executes user code in the ERTS.

OTP

Open Telecom Platform - a framework that provides an abstraction of common uses of / interactions with
certain types of processes. It provides modules and behaviours that represent standard implementations of
common practices like process supervision, message passing, spawning tasks, etc.

ERTS

Erlang Run Time System - the system that contains functionality necessary to run the Erlang system.

OS

Operating System - system software that manages computer hardware and software resources, and
provides common services for computer programs.

Websockets

A computer communications protocol, providing a simultaneous two-way communication channel
over a single TCP connection.

TCP

Transmission Control Protocol - one of the main protocols of the Internet protocol suite. TCP provides
reliable, ordered, and error-checked delivery of a stream of bytes between applications over a IP network.

HTTP

HyperText Transfer Protocol - an application layer protocol in the Internet protocol suite.
It is used for transferring data between clients and servers.

DoS

Denial of Service - An attack where the perpetrator seeks to make a machine or network resource unavailable
to its intended users

%---


\bibliography{plan_refs}
\end{document}
\end{article}